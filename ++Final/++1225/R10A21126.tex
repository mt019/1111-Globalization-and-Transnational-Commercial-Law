% 快點寫功課!!!
\documentclass[]{article}
\makeatletter\if@twocolumn\PassOptionsToPackage{switch}{lineno}\else\fi\makeatother

  
\usepackage{amsmath,amsfonts,amsbsy,amssymb,tabulary,graphicx,times,caption,fancyhdr}
\usepackage[utf8]{inputenc}
\usepackage[paperheight=10in,paperwidth=6.5in,margin=2cm,headsep=.5cm,top=2.5cm,headheight=1cm]{geometry}
\renewenvironment{abstract} {\vspace*{-1pc}\trivlist\item[]\leftskip\oupIndent\hrulefill\par\vskip4pt\noindent\textbf{\abstractname}\mbox{\null}\\\relax}{\par\noindent\hrulefill\endtrivlist} 
\linespread{1.13} \date{}
\captionsetup[figure]{labelfont=sc,skip=1.4pt,aboveskip=1pc}
\captionsetup[table]{labelfont=sc,skip=1.4pt,labelsep=newline}



\makeatletter\def\oupIndent{1pt}
\def\author#1{\gdef\@author{\hskip-\dimexpr(\tabcolsep)\hskip\oupIndent\parbox{\dimexpr\textwidth-\oupIndent}{\centering\bfseries#1}}}
\def\title#1{\gdef\@title{\centering\bfseries\ifx\@articleType\@empty\else\@articleType\\\fi#1}}
\let\@articleType\@empty \def\articletype#1{\gdef\@articleType{{\normalfont\itshape#1}}}
\fancypagestyle{headings}{\fancyhf{}\fancyhead[C]{\RunningHead}\fancyhead[R]{\thepage}}\pagestyle{headings}
\makeatother

  


\tolerance=5000
%%%%%%%%%%%%%%%%%%%%%%%%%%%%%%%%%%%%%%%%%%%%%%%%%%%%%%%%%%%%%%%%%%%%%%%%%%
% Following additional macros are required to function some 
% functions which are not available in the class used.
%%%%%%%%%%%%%%%%%%%%%%%%%%%%%%%%%%%%%%%%%%%%%%%%%%%%%%%%%%%%%%%%%%%%%%%%%%
\usepackage{
  url,
multirow,morefloats,floatflt,cancel,tfrupee}
% \usepackage[hyphens]{url}
\usepackage{hyperref} 



\makeatletter


\AtBeginDocument{\@ifpackageloaded{textcomp}{}{\usepackage{textcomp}}}
\makeatother
\usepackage{colortbl}
\usepackage{xcolor}
\usepackage{pifont}
\usepackage[nointegrals]{wasysym}
\usepackage{enumitem}

\urlstyle{rm}
\makeatletter

%%%For Table column width calculation.
\def\mcWidth#1{\csname TY@F#1\endcsname+\tabcolsep}

%%Hacking center and right align for table
\def\cAlignHack{\rightskip\@flushglue\leftskip\@flushglue\parindent\z@\parfillskip\z@skip}
\def\rAlignHack{\rightskip\z@skip\leftskip\@flushglue \parindent\z@\parfillskip\z@skip}

%Etal definition in references
\@ifundefined{etal}{\def\etal{\textit{et~al}}}{}


%\if@twocolumn\usepackage{dblfloatfix}\fi
\usepackage{ifxetex}
\ifxetex\else\if@twocolumn\@ifpackageloaded{stfloats}{}{\usepackage{dblfloatfix}}\fi\fi

\AtBeginDocument{
\expandafter\ifx\csname eqalign\endcsname\relax
\def\eqalign#1{\null\vcenter{\def\\{\cr}\openup\jot\m@th
  \ialign{\strut$\displaystyle{##}$\hfil&$\displaystyle{{}##}$\hfil
      \crcr#1\crcr}}\,}
\fi
}

%For fixing hardfail when unicode letters appear inside table with endfloat
\AtBeginDocument{%
  \@ifpackageloaded{endfloat}%
   {\renewcommand\efloat@iwrite[1]{\immediate\expandafter\protected@write\csname efloat@post#1\endcsname{}}}{\newif\ifefloat@tables}%
}%

\def\BreakURLText#1{\@tfor\brk@tempa:=#1\do{\brk@tempa\hskip0pt}}
\let\lt=<
\let\gt=>
\def\processVert{\ifmmode|\else\textbar\fi}
\let\processvert\processVert

\@ifundefined{subparagraph}{
\def\subparagraph{\@startsection{paragraph}{5}{2\parindent}{0ex plus 0.1ex minus 0.1ex}%
{0ex}{\normalfont\small\itshape}}%
}{}

% These are now gobbled, so won't appear in the PDF.
\newcommand\role[1]{\unskip}
\newcommand\aucollab[1]{\unskip}
  
\@ifundefined{tsGraphicsScaleX}{\gdef\tsGraphicsScaleX{1}}{}
\@ifundefined{tsGraphicsScaleY}{\gdef\tsGraphicsScaleY{.9}}{}
% To automatically resize figures to fit inside the text area
\def\checkGraphicsWidth{\ifdim\Gin@nat@width>\linewidth
	\tsGraphicsScaleX\linewidth\else\Gin@nat@width\fi}

\def\checkGraphicsHeight{\ifdim\Gin@nat@height>.9\textheight
	\tsGraphicsScaleY\textheight\else\Gin@nat@height\fi}

\def\fixFloatSize#1{}%\@ifundefined{processdelayedfloats}{\setbox0=\hbox{\includegraphics{#1}}\ifnum\wd0<\columnwidth\relax\renewenvironment{figure*}{\begin{figure}}{\end{figure}}\fi}{}}
\let\ts@includegraphics\includegraphics

\def\inlinegraphic[#1]#2{{\edef\@tempa{#1}\edef\baseline@shift{\ifx\@tempa\@empty0\else#1\fi}\edef\tempZ{\the\numexpr(\numexpr(\baseline@shift*\f@size/100))}\protect\raisebox{\tempZ pt}{\ts@includegraphics{#2}}}}

%\renewcommand{\includegraphics}[1]{\ts@includegraphics[width=\checkGraphicsWidth]{#1}}
\AtBeginDocument{\def\includegraphics{\@ifnextchar{\ts@includegraphics}{\ts@includegraphics[width=\checkGraphicsWidth,height=\checkGraphicsHeight,keepaspectratio]}}}

\DeclareMathAlphabet{\mathpzc}{OT1}{pzc}{m}{it}

\def\URL#1#2{\@ifundefined{href}{#2}{\href{#1}{#2}}}

%%For url break
% \def\UrlOrds{\do\*\do\-\do\~\do\'\do\"\do\-\do\/}%
% \g@addto@macro{\UrlBreaks}{\UrlOrds}
\PassOptionsToPackage{hyphens}{url}\usepackage{hyperref}



\edef\fntEncoding{\f@encoding}
\def\EUoneEnc{EU1}
\makeatother
\def\floatpagefraction{0.8} 
\def\dblfloatpagefraction{0.8}
\def\style#1#2{#2}
\def\xxxguillemotleft{\fontencoding{T1}\selectfont\guillemotleft}
\def\xxxguillemotright{\fontencoding{T1}\selectfont\guillemotright}

\newif\ifmultipleabstract\multipleabstractfalse%
\newenvironment{typesetAbstractGroup}{}{}%

%%%%%%%%%%%%%%%%%%%%%%%%%%%%%%%%%%%%%%%%%%%%%%%%%%%%%%%%%%%%%%%%%%%%%%%%%%



% \usepackage[numbers,sort&compress]{natbib}

\usepackage[sorting=none, citestyle=verbose-inote,backref=true,ibidtracker=context,mincrossrefs=99,backend=biber, 
url = false,
doi = false, isbn=false,]{biblatex}
% Document
% https://www.emse.fr/~picard/files/biblatex.pdf
% Cheatsheet
% https://ftp.ntou.edu.tw/ctan/info/biblatex-cheatsheet/biblatex-cheatsheet.pdf

\addbibresource{R10A21126.bib}

\usepackage{hyperref}
\usepackage{lipsum} 

\usepackage[T1]{fontenc}

%%%%%%%%%%%%%%%%%%%%%%%%%%%%%%%%%%%%%%%%%%
% Feature enabled:
%full-reference: true
%toc: yes
%%%%%%%%%%%%%%%%%%%%%%%%%%%%%%%%%%%%%%%%%%
\makeatletter\@ifundefined{tableofcontents}{\usepackage{typeset-custom-toc}}{}\makeatother
\usepackage{etoolbox}

% defines the paragraph spacing
\setlength{\parskip}{0.5em}




\begin{document}


% \nocite{*}


\title{International Law and Taxation}
\author{\textbf{\fontsize{14pt}{16.4pt}\selectfont{YIFAN WANG}}~\\\normalsize\normalfont {College of  Law \unskip, National Taiwan University}~\\{\normalsize\normalfont  E-mail: R10A21126@ntu.edu.tw}}
\def\RunningHead{{International Law and Taxation}}

\maketitle 


\begin{abstract}
This paper reviews the literature and identifies the relationship between the tax system and state sovereignty, especially in the age of globalization.
The paper also tries to sort out the interaction between taxation and international commercial activities.
In addition, the paper provides an outlook on the coming development of the international tax system.
  
It calls for a multi-level and multilateral international tax regime that provides a friendly and stable environment for globalized economic transactions, contributes to net-zero carbon emissions, and maintains the tax base for governments around the world to achieve tax justice.

\end{abstract}

\def\keywordstitle{Keywords}

\smallskip\noindent\textbf{Keywords: }{Tax, Sovereignty, Globalization, International Tax Governance, International Tax Tribunal} 


\clearpage

\tableofcontents


\pagebreak
\section{Introduction}

With the globalization of economies, international trade brings profit all over the world, making it possible for people and businesses to make the wealth distributed and avoid tax in their home countries. In 2010, the Foreign Account Tax Compliance Act (FATCA) was enacted by Congress of the U.S. to target non-compliance by U.S. taxpayers using foreign accounts. Since then, the global regime for taxation has seen a trend of cooperation and integration, rather than competition. The mechanism of tax competition, harmonization, or coordination is complicated. 

There is expected to be a globally efficient tax structure for international business. Only coordination can help to prevent tax evasion and avoidance. Jurisdictions over the world need to work together to provide long-term certainty to the international tax system. The tax regimes must have a better idea of globalization since tax is one of the most important areas impacted by it. 

% However, there are a few questions. What is the relationship between taxation and international trade? What is the relationship between tax treaties and free trade agreements or international investment agreements? Since there are numerous international tax disputes, why can they not be dealt with in a similar way to multilateral international organizations or agreements? Is it possible to establish an international tax tribunal?

\section{Tax and Sovereignty}
Taxation is a very special area. The tax system is highly related to state sovereignty, the idea of which is based on the principle of non-intervention. However, it is possible that although states seek to protect their \textit{de jure} sovereignty (their right to rule), the resulting tax competition limits their discretionary space to pursue the fiscal policies they want\footcite[216]{Dietsch2016}.
Therefore, according to the analysis of Ralf Michaels, transactional law, as a theory of law beyond the state, is attractive\footcite[300]{Michaels2013}.

\subsection{Tax as Last Bastion of Westphalian Sovereignty}

Tax is considered a symbol of the sovereignty of the State. The power of taxation is referred to as one of the inherent powers of the government and an inherent attribute of sovereignty, which means that it exists simply because the State exists\footcite{Mabel2022}. 


% Small economies and capital owners have powerful voices. And that voice includes the argument that every country has the sovereign right to set their tax rates as they see fit, which is a significant argument in a world stuck in 1648 Westphalia\footcite[177]{Dietsch2016}. 

It has been argued that taxation seems to be the last bastion of entrenched perceptions of national sovereignty, an undisputed cornerstone of the independent and authoritative government, the undeniable prerogative of national policy-makers in the face of growing global economic integration\footcite{Christensen2016}. 

Many of the multinational agreements that states enter into based on their legal sovereignty at the international level will curtail Westphalian sovereignty. Trade agreements like the WTO agreements or NAFTA that come with dispute procedures are examples of international arrangements that compromise Westphalian sovereignty\footcite[2110]{Dietsch2011}. 
However, the inherent belief in preserving sovereignty is reflected in the fact that when a country joins an international organization or signs a trade agreement, the sovereignty of taxation is preserved by the country.  Even the EU countries retain their tax sovereignty to a great extent, only with the EU overseeing national tax rules in some areas, particularly concerning EU business and consumer policies. As a result, countries have the sovereign right to design their tax systems as they see fit. 
% In fact, the tax systems would b

\subsection{Tax Avoidance and Evasion}
% tax compliance, avoidance, and evasion
% Anti-Double Taxation and Avoidance of Double Non-Taxation
At the level of economic globalization, it has been common for the rich and businesses to take undue advantage of the loophole when the tax regulations differ among jurisdictions, especially in terms of income tax. Complicated transaction structures are established worldwide to lower the effective tax rate. 

Multinationals are capable to do tax planning. Researchers have shown that larger firms have a greater tendency to avoid income taxes and commit tax evasion\footcite{Hoxhaj2022}. They can analyze the tax regulations of jurisdictions around the world and plan to ensure that their business structures and profits are well arranged to allow them to pay the lowest taxes possible, probably by shifting their profits from high-tax countries to low-tax jurisdictions and tax havens. 
The business structures might be designed deliberately rather than following the normal business model, only for lower taxes.

The phenomenon has been specifically discussed by The Organization for Economic Co-operation and Development (OECD). OECD summarises the tax planning strategies used by multinational enterprises to avoid paying tax as Base Erosion and Profit Shifting (BEPS). The tax planning strategies are described by OECD as "exploiting gaps and mismatches in tax rules". It is believed that developing countries' higher reliance on corporate income tax means they suffer from BEPS disproportionately\footcite{OECD_BEPS}.

Taking one of the world's largest technology companies as an example, Apple operated a tax structure called Double Irish With a Dutch Sandwich tax structure. The planning involves sending profits first through one Irish company, then to a Dutch company, and finally to a second Irish company headquartered in a tax haven\footcite{Snowden2017}.
It is reported that many multinationals reduce their global taxation liabilities through the use of similar taxation regimes\footcite{2022}.


\subsection{Tax Competition Hurts Sovereignty}

With the globalization of economies, it is important to have a different view on the tax regime. It has been argued that it is because countries are struggling to preserve their Westphalian sovereignty that arbitrage is possible and leads to the erosion of domestic sovereignty (states' fiscal policy discretion or effectiveness)\footcite[2110]{Dietsch2011}, and it is claimed that states ought to have effective or positive sovereignty, focusing on strengthening the democratic character of collective will-formation\footcite[225]{Dietsch2016}. The tax regimes must have a better idea of globalization since tax is one of the most important areas impacted by it. 


% It is because countries struggle to preserve their Westphalian sovereignty that arbitrage becomes possible and the erosion of domestic sovereignty results.





\section{International Tax Regime and Dispute Resolution}

\subsection{Business, Tax, and Disputes}

There have been researches on the relationship between domestic taxes (labor taxes \footcite{Adarov2021}  and corporate taxes \footcite{Holzner2021} ) and international trade industries. It is obvious that (income and non-income) taxation largely impacts the decisions of businesses. The tight relationship between international business and taxation has also brought about tax disputes between businesses and governments.


As more attention is paid to environmental issues, countries are making progress in imposing environmental taxes on economic activities. One of the most important topics is the carbon tax, which aims to induce industries, businesses, and households to reduce their carbon emissions, thereby achieving carbon neutrality and slowing down global warming. International trade will certainly be affected by this trend as well.

EU has taken the lead. One of the measures is the Carbon Border Adjustment Mechanism (CBAM), which would place a carbon tariff on electricity, cement, aluminum, fertilizer, and iron and steel products imported from outside the EU, thus leveling the playing field for European producers and avoiding carbon leakage.
There may be a lot of concerns about the difference in environmental tax regulations. It is possible that environmental tax could be a significant reason for international tax disputes, besides income tax.

% Not only the difference in domestic tax regulations can be taken undue advantage of, but the tax treaties could also be abused.

% The intricacies of tax treaties are intended to avoid double taxation, but they give large corporations room for tax avoidance. Taxpayers can keep their profits parked in tax havens through layers of transaction arrangements.

\subsection{BEPS Actions, MAPs, and Arbitration}
In terms of income tax, OECD originated the Action Plan on tax base erosion and profit shifting (BEPS) in 2015. In 2021, 136 members joined a new two-pillar plan to reform international tax rules and ensure that multinational enterprises pay a fair share of tax wherever they operate. This is exactly an example of 'unbound' jurisdiction described in Krisch's paper\footcite{Krisch2022}, where coordination could hopefully be achieved. 

% \subsection{}

With BEPS Action 14, Article 25 of the OECD Model Tax Convention (OECD, 2014) provides a mechanism of dispute resolution -- Mutual Agreement Procedures (MAPs). MAP should be helpful to prevent disputes in advance. This action in the tax treaty also enables taxpayers to present their dispute cases to the competent authorities (CAs) of the Contracting Parties. The CAs shall consult together and endeavor to resolve the cases in a timely, effective, and efficient manner. MAPs could also help to eliminate income double taxation, and to ensure tax certainty for taxpayers if the procedures are implemented.

There is also a provision for an arbitration mechanism to resolve disputes that may arise between the two countries relating to the interpretation or application of tax treaties.
Under Article 25, if the competent authorities of the two countries are unable to resolve a dispute through mutual agreement, either country may request that the dispute be submitted to arbitration. 

% The arbitration panel consists of three members, one chosen by each country and a third member who is chosen by the other two members or appointed by an independent organization, such as the International Centre for Settlement of Investment Disputes (ICSID).

The arbitration panel has the authority to interpret the provisions of the tax treaty and to decide on the specific actions that the countries must take to resolve the dispute. The decision of the arbitration panel is final and binding on the two countries, and they are required to implement the panel's decision.

\subsection{Tax Treaty and Non-Tax Treaty}

Tax treaties produce mechanisms to share the income tax base between countries, promote cooperation in the application of the treaties, and resolve disputes, while international investment law governs foreign direct investment and the resolution of disputes between foreign investors and host states. They appear to be two separate legal regimes. Nevertheless, they largely overlap and interact. Moreover, tax is examed as a potential barrier to investment and cross-border trade \footcite{Chaisse2016}.

Specifically, taxation may be a matter of investment law. Host States' conduct to address tax erosion by investors can be restricted by substantive obligations derived from International Investment Agreements (IIAs). If isn't excluded, taxation is covered by investment treaties. The exceptions regime is complex and can be categorized into 7 main types\footcite[16]{Chaisse2016}:

\begin{enumerate}[itemsep=0em]
  \item General exclusion
  \item  Limited exclusion in relation to NT and MFN standards
  \item  Limited exclusion in relation to fair and equitable treatment
  \item  Limited exclusion based on the distinction between direct and indirect taxes
  \item Tax veto to expropriation case 
  \item Priority of taxation treaties over IIAs 
  \item Complex combination of exceptions within
  exclusion.
\end{enumerate}
The purpose of the carve-out clause is to ensure that the host state retains its sovereignty to determine tax policy\footcite[13]{Chaisse2016}. However, as discussed before, this may not benefit the overall domestic sovereignty.


    
% \subsection{International Trade and Environmental Taxes}


    

% \subsection{Investor-state arbitration and tax disputes}
% Main instrument: Mutual Agreement Procedures (MAPs).

\section{International Tax Tribunal}
\subsection{Review the Existing Dispute Resolution Systems}


% Concerning future trends in dispute resolution, Bookman has helpfully pointed out that the proliferation of international commercial courts undermines three conventional narratives. The first conventional narrative is that there is a ‘race to the top’ for the most efficient dispute resolution mechanism; second, litigation and arbitration are alternative mechanisms; and finally, parties prefer arbitration to litigation in international commercial disputes.


Inspired by Bookman's opinion \footcite{Bookman2019} on the International Commercial Court and the trends in dispute resolution, this paper argues that an International Tax Tribunal 
\footnote{Notice that the International Tax Tribunal is proposed here as an international court rather than a domestic court.} 
could be a good option for resolving tax disputes for the following reasons:

\begin{enumerate}
  \item The existing international tax dispute resolution system mainly consists of domestic courts and arbitration mechanisms (only for income tax disputes with regards to the tax treaties), to which the jurisdictions are limited.
  \item The domestic courts are to some extent expressions of state sovereignty and may lack expertise in international tax disputes or the overall international tax regimes. The decisions of the domestic courts are not necessarily binding on the other countries involved in the disputes. This can make it difficult for the courts to enforce the decisions.
  \item The arbitration mechanism from Article 25 of the OECD Model Tax Convention is only likely to function in the case of income tax treaties. In general, arbitration is a final and binding process, and the decisions of the arbitrators are typically not subject to appeal. Although it is typically a faster and more efficient process than litigation, the participation and the autonomy of the taxpayer are limited since the process is between the tax authorities of the two countries about the interpretation or application of the treaty provisions.
  \item Parties (especially the taxpayers) may prefer litigation for its greater transparency and accountability
  but are more
  comfortable with arbitration by reason of its neutrality and flexibility
  .
  However, it is litigation (rather than arbitration) could stimulate the coherent and transparent development of international tax law.

  \item  International tax cases can involve the tax interests of multiple countries, so a multilateral dispute resolution solution is needed.
\end{enumerate}

The existing international tax dispute resolution systems can have limitations in terms of their jurisdiction, accessibility, binding authority, and resources, which can make it challenging for individuals and businesses to resolve tax disputes and can undermine the effectiveness and fairness of international tax regimes. Therefore, a more effective international tax tribunal is expected, which will undoubtedly require compromises on national sovereignty but will be beneficial to the overall international industrial, economic, and legal development.

\subsection{Features of the International Tax Tribunal} 

With the emergence of international tax disputes between countries and between taxpayers and countries, a multi-level and multilateral International Tax Tribunal could be a competitive alternative regime for international tax dispute resolution with the following features:

\begin{enumerate}
  \item Jurisdiction: The tribunal could have broad jurisdiction to hear cases involving tax disputes between countries and between taxpayers and countries. The cases can be either income tax disputes or non-income tax disputes (e.g. international environmental tax disputes). It could also have the authority to interpret income tax treaties and other international tax agreements or non-tax agreements concerning tax issues.
  \item Composition: The tribunal could be composed of judges with expertise in tax law and international law, who could be appointed from a pool of qualified candidates from around the world. The tribunal could also include representatives from a range of countries, in order to reflect the diversity of the international tax system and respect the tax interests of each country.
  \item Litigation: There would be a specialized court as an independent body with a panel of judges who are experts in international tax law. The judges could be appointed by a panel of international tax experts, rather than by individual countries, to ensure the independence and impartiality of the court. There would also be the appeal body to which parties could have the right to appeal a decision of the court.
  \item Alternative Resolution: As a one-stop dispute resolution center, the tribunal could provide alternative dispute resolution options, including arbitration and mediation, in order to accommodate the needs and preferences of the parties involved.
  \item Binding authority: The decisions of the tribunal could be binding on the countries and taxpayers involved in the dispute. This could help to ensure that the tribunal's decisions are implemented and that they have a deterrent effect on future tax disputes.
  \item Resources: The tribunal could be funded by member countries and could hopefully have sufficient resources to ensure that it can operate effectively and efficiently.
  
\end{enumerate}

Overall, an International Tax Tribunal could provide a fair, impartial, and effective forum for resolving tax disputes at the international level. It could help to ensure the consistent and effective application of international tax regulations and could contribute to the stability and predictability of international tax regimes.

% \subsection{Multi-level and Multilateral} 
% multi-level and multilateral 
% \subsection{Improving the Process for Resolving International Tax Disputes} 


% Foster cooperation between countries and multinational corporations, by providing a forum for resolving disputes in a mutually beneficial manner.

% Overall, an international tax court could provide a valuable forum for resolving cross-border tax disputes and promoting cooperation in the international tax system.

% Overall, the international tax court would provide a specialized forum for resolving cross-border tax disputes fairly and efficiently, promoting consistency and predictability in international tax law and fostering cooperation between countries and multinational corporations.

\section{Conclusion} 


% This article reviews the literature and identifies the importance of the tax system to state sovereignty.
% The paper also tries to sort out the interaction between taxation and international trade as well as investment.
% In addition, the paper provides an outlook on the coming development of the international tax system.

% It calls for a multi-level and multilateral international tax regime that provides a friendly and stable environment for globalized economic transactions, contributes to net-zero carbon emissions, and maintains the tax base for governments around the world to achieve tax justice.
The traditional regimes for tax regulations are meant to be updated with the idea of globalization. The traditional ideal of state sovereignty can be a barrier against globalization and states' tax interests.
Only coordination can help to prevent international tax evasion and avoidance, and eliminate disputes. Sovereignty could be preserved by compromising and mutual agreements. 


With the trend of global governance of tax, the existing international tax dispute resolution is considered to be insufficient and not effective enough to deal with complicated cross-border tax disputes among states and taxpayers.
This paper proposes an international tax tribunal, where disputes could be solved with international consensus, as a competitive alternative system.
There is expected to be a multi-level and multilateral international tax dispute resolution system that provides a friendly and stable environment for globalization for resolving international tax disputes and benefiting tax interests for governments around the world to achieve tax justice. The development of international tax legislation could also benefit from the practice of the Tribunal.


\pagebreak

% \begin{refcontext}[sorting=nyt]
% \emergencystretch 1.5em
% \sloppy

% \setcounter{biburllcpenalty}{7000}
% \setcounter{biburlucpenalty}{8000}
\Urlmuskip=0mu plus 1mu\relax
% \biburllcpenalty
\setcounter{biburllcpenalty}{7000}
\setcounter{biburlucpenalty}{8000}
\printbibliography
  % \end{refcontext}
  % \printbibheading
  % \printbibliography[type=book,heading=subbibliography,title={Book Sources}]
  % \printbibliography[nottype=book,heading=subbibliography,title={Other Sources}]

\end{document}
